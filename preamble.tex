%======================================================================
%	MAKEFILE
%======================================================================

% ----------------------------------------: DOCUMENT DEFINITION
\ifthenelse{\boolean{PrintVersion}}{
	\usepackage[top=1in,bottom=1in,left=0.75in,right=1.25in]{geometry}
}{
	\usepackage[top=1in,bottom=1in,left=0.75in,right=1.25in]{geometry}
}

\usepackage[utf8]{inputenc}


% ----------------------------------------: MATHEMATICS
\usepackage{amssymb}
\usepackage{amsmath}
\usepackage{amstext}
\usepackage{esvect}		% Vector arrow symbols
\usepackage{siunitx}	% Pretty scientific notations


% ----------------------------------------: FORMATTING PACKAGES
\usepackage{fancyhdr}	% Page header/footer support
\usepackage{enumitem}	% Enumerations environment
\usepackage{float}		% Floating figures
\usepackage{longtable}	% Multipage tables
\usepackage{listings}	% Formats code using the lstlisting environment
\usepackage{tcolorbox}	% Beamer-style boxes
\usepackage{xargs}
\usepackage{titlesec}	% Section format customization
\usepackage{array}		% Table support
\usepackage{cite}		% Citations within the document
\usepackage{tabularx}
\usepackage{booktabs}
\usepackage{xtab}
\usepackage{lastpage}

% Color specification support
\usepackage{xcolor}

% Strikethrough support
\usepackage[normalem]{ulem}

% Todo notes support
\usepackage[colorinlistoftodos,prependcaption,textsize=tiny]{todonotes}

% Figure caption support
\usepackage[labelfont=bf, skip=2pt]{caption}


% ----------------------------------------: MISCELLANEOUS
\usepackage{lipsum}		% Generate lorem ipsum dummy text


% ----------------------------------------: SETTINGS
\graphicspath{ {figures/} }		% Set graphics path
\sisetup{unitsep=\cdot}			% Scientific notation seperator

% Headers and footers
\pagestyle{fancy}
\fancyhf{}
\lhead{Umbrella Corp.}
\chead{}
\rhead{Seaborn}
\lfoot{2021}
\cfoot{}
\rfoot{Page \thepage}


% ----------------------------------------: FORMAT DEFINITIONS
% Define specific note formats
\newcommandx{\unsure}[2][1=]{\todo[linecolor=red,backgroundcolor=red!25,bordercolor=red,#1]{#2}}
\newcommandx{\explain}[2][1=]{\todo[linecolor=blue,backgroundcolor=blue!25,bordercolor=blue,#1]{#2}}

% Define section formats
\titleformat{\section}{\normalfont\large\bfseries}{\S\thesection}{1em}{}[{\titlerule[0.6pt]}]
\titleformat{\subsection}{\normalfont\normalsize\bfseries}{\thesubsection}{1em}{}
\titleformat{\subsubsection}{\normalfont\normalsize\itseries}{\thesubsubsection}{1em}{}

% Define table column formats
\newcolumntype{L}[1]{>{\raggedright\let\newline\\\arraybackslash\hspace{0pt}}m{#1}}
\newcolumntype{C}[1]{>{\centering\let\newline\\\arraybackslash\hspace{0pt}}m{#1}}
\newcolumntype{R}[1]{>{\raggedleft\let\newline\\\arraybackslash\hspace{0pt}}m{#1}}

% Does nothing, but defines the command so the print-optimized version
% will ignore \href tags which are defined later
\newcommand{\href}[1]{#1}

% Define hyperlink behavior
\usepackage{hyperref}
\hypersetup{
	colorlinks=true,		% false: boxed links; true: colored links
	linkcolor=blue,			% internal links
	citecolor=green,		% links to bibliography
	filecolor=magenta,		% file links
	urlcolor=cyan			% external links
}
\ifthenelse{\boolean{PrintVersion}}{
\hypersetup{
	citecolor=black,
	filecolor=black,
	linkcolor=black,
	urlcolor=black}
}{}

% Force longer URLs to wrap-around
\expandafter\def\expandafter\UrlBreaks\expandafter{\UrlBreaks%  save the current one
	\do\a\do\b\do\c\do\d\do\e\do\f\do\g\do\h\do\i\do\j%
	\do\k\do\l\do\m\do\n\do\o\do\p\do\q\do\r\do\s\do\t%
	\do\u\do\v\do\w\do\x\do\y\do\z\do\A\do\B\do\C\do\D%
	\do\E\do\F\do\G\do\H\do\I\do\J\do\K\do\L\do\M\do\N%
	\do\O\do\P\do\Q\do\R\do\S\do\T\do\U\do\V\do\W\do\X%
	\do\Y\do\Z}



% Fin.
